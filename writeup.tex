\documentclass[11pt,letterpaper]{article}
\usepackage{naaclhlt2010}
\usepackage{times}
\usepackage{latexsym}
\usepackage{url}
\usepackage{graphicx}
\usepackage{rotating}
\usepackage{multirow}
\usepackage{float}
\usepackage{textcomp}
\usepackage{enumerate}
\setlength\titlebox{6.5cm}
%\renewcommand{\baselinestretch}{0.96}


\title{Fast Intersection}

\author{
	Adam Gerber \\
	{\tt adam.gerber@jhu.edu}\\
	Johns Hopkins University\\
	3400 North Charles Street \\
	Baltimore, MD 21210, USA\\
	\And 
	{Timur Sherif} \\
	{\tt tsherif1@jhu.edu}\\
	Johns Hopkins University\\
	3400 North Charles Street \\
	Baltimore, MD 21210, USA
}

\begin{document}
\maketitle

\begin{abstract}
This paper proposes a novel set intersection algorithm
especially effective for sparse sets over large range
still maintains state of the art performance for dense sets over small range
\end{abstract}

\section{Introduction}
The set intersection problem of two sets $A$ and $B$ is to find the set that contains all elements of $A$ that also appear in $B$, but no other elements. 

Formally: 
\[x \in A \cap B \iff x \in A \land x \in B\]

This definition can be expanded to the intersection of multiple sets where the intersection is the set that contains all elements of one set that also appear in all other sets, but no other elements. This problem is often encountered in the context of the evaluation of relational queries to databases and search engines. A number of different approaches have been taken to this problem. Demain, et al [CITATION NECESSARY] proposed an intersection algorithm which intersects multiple sets in parallel. Barbay, et al propose using many algorithms with certain improvements and a type of interpolation search. They compared the performance between the various algorithms. The Baeza-Yates algorithm [CITATION NECESSARY] attempts to minimize the time of the search by using a recursive solution (on two sorted sets): using a binary search, search for the median element of the smaller set in the larger set and divide the problem into two subproblems. If the median of the smaller set exists in the larger set, add it to the intersection. Each of these algorithms use a specific search algorithm and then minimize the search space in some way. Udomchaiporn et al, proposed a different approach to the intersection of sorted sets using a comparison and elimination approach. Their algorithm repeatedly finds the maximum value of the first elements of each set and the minimum of the last elements of each set to limit the range. The common minimum or maximum values become elements of the intersection set.

NEED TO CHANGE: In this article, we propose a new approach to the intersection problem by exploiting the properties of bit vectors to eliminate comparisons of ranges of numbers that do not overlap.


(We maintain multiple indices over the set which allow us to do in-order traversals over the values of each column. )

The article is structured as follows: In section 2, we present related works, in section 3 . . . . . . . . . . . 



\section{Related Work}
As described above, there are many approaches to the intersection problem. Some of those more relevant to our approach are described in detail below. Demain, et al. proposed the Adaptive algorithm which searches for an element of a particular set from all other sets using a combination of ``galloping search`` and binary search. Essentially, the algorithm ``gallops`` in parallel through all the sets from both the low and high sides to find eliminators (elements possible in the intersection). ``Galloping consists of doubling the jump in position each iteration, until it overshoots the current eliminator which will be on the low side. Upon overshooting, the other parallel processes pause while the overshooter does a binary search to find the largest eliminatable element and chooses the next higher element as the new eliminator.`` This algorithm had applications to search engines, but many of the techniques used create negative impacts such as the high complexity of adapting to numerous sets of data. Barbay, et al.'s Sequential essentially uses a similar approach except for its use of the galloping search as described above. This performs less comparisons than the Adaptive algorithm on average. The Baeza-Yates algorithm takes the median element of the smaller of two sorted sets and uses a binary search to find it in the larger set. If the element is in the larger set, it is added to the intersection set. This algorithm then recursively solves the instances of each pair of subsets. Even though this algorithm is relatively fast, this causes the memory space used by this algorithm to be greater than any of the other algorithms. The Search-Free algorithm described by Udomchaiporn, et al. takes a different approach to intersecting sorted sets. It finds a maximum value of the first element of each set and the minimum value of the last element of each set. These two values become a range of possible elements in the intersection set. This process continues until the minimum and maximum are equal. In the intermediary step, if the the minimum or maximum element in one set matches the minimum or maximum element in all other sets, this element is added to the intersection set. The downside to this approach is that in sets that have large sections of elements that do not match, the algorithm must set all of these elements to null. This process proceeds in linear time with the size of the non-matching elements.
All of these approaches have benefits when dealing with certain situations, but they essentially must compare elements one at a time. Therefore we propose an approach that compares elements in blocks to overcome this major limitation.


Demaine, et al, [CITATION] instead of focusing on the worst-case complexity of the intersection of sorted sets, considered the idea of complexity that depends on a particular instance. They developed the idea of a Proof (characterized in the paper) that a given set is the correct answer. The adaptive algorithms they present make no a priori assumptions and their running times are within a constant factor of optimal "with respect to a natural measure of the difficulty of the instance." 
Baeza-Yates' et al algorithm [CITATION] was significantly simpler: for two sorted sets A and B, where B is small,  search the median of B in A. If found, add the element to the intersection set. Now, the problem is divided into two subproblems (the parts of the sets to the right and left of the median of B). Using this algorithm, they recursively solve both parts. 
Barbay, et al, proposed a variant form of the Baeza-Yates algorithm which performs the intersection of more than two sorted arrays without sorting the intermediary results. According to them, this variant is significantly faster than the original algorithm in terms of number of comparisons performed and CPU time. They also tried to introduce value-based search algorithm to improve other intersection algorithms.
Udomchaiporn, et al proposed a different approach to intersection of sorted sets by using a comparison and elimination approach to avoid using any search algorithms. This algorithm finds the maximum value of each first element of all sets and the minimum value of each last element in all sets. This limits the range of possible elements in the intersection set. If any minimum or maximum value is common to all sets, it gets added to the intersection set. This procedure is repeated until all values are eliminated.
While all of these algorithms �������������..

Context:
abstract v. definite framework

prolog or similar solver in which the algorithm provides the requestor with an iterator which when queried 


metric: ratio of operations to results returned

\section{Method}
\subsection{Context}

While most papers have described algorithms abstractly, with only a slight reference to their actual application, we propose this algorithm in the framework of a solver such as Prolog or a similar type of solver. We explore the algorithm in the context of returning an intersection, but it could just as easily provide the requestor with an iterator, which when queried could obtain any records associated with the key. A simple example of this would be a query on two tables to find all products that have a coupon (figure X). In this case, the algorithm would return an iterator over pids within the intersection: 

\begin{enumerate}[(a)]
\item	1: (HDD x, a)
\item	2: (HDD y, b), (HDD y, c)
\item	4: (RAM a, d)
\end{enumerate}


\subsection{Definitions}

We introduce a tree structure for storing $m$-bit integer keys,
where $m$ must be of the form $m=Di$ for some small
integers $D, i,$ and $n$, such that $i=2^n$.  For example,
to represent 32-bit integers, one such assignment would be
$n=3, i=2^n=8,$ and $D=4$.  As such, it is possible to segment
every $m$-bit integer into $d$ contiguous $i$-bit sequences.
We will use the notation $sig_i(key, j)$ to denote a $key$'s
$j^{th}$ most significant $i$-bit sequence.  In other words,
for any key $k_q$ in the range $2^{Di}$, the following serves as
an identity, where $\oplus$ is the append operator:

\[ k_q = sig_i(k_q, 0) \oplus sig_i(k_q, 1) \oplus ... \oplus sig_i(k_q, D-1) \]

%TODO finish this formal definition of the tree
Formally, a tree $T$ consists of the a set of a finite set of nodes,
indexed by a hash-table, {\Huge TODO}
a , = consists of a this structure consists of represents
Nodes are tuples of the form $(c, m, p)$, where $c \in [0, 2^{2^i}]$
indicates the current bit configuration, $m$ the number of leaf nodes
(equivalent to the number of unique keys that have been inserted
into that node's subtrees), and $p$ the total number of non-leaf nodes
beneath that node.

% TODO - Connections to Van Emde Boas trees & suffix trees

\begin{figure}[t]
\center
  %\fbox{
\includegraphics[width=75mm]{example-vector.pdf}
%}
\caption{An example bit-vector for $i=4$.}
\label{fig:example-vector}
\end{figure}

\subsubsection{Structure}
For a tree T over the universe $0$ to $2^m - 1$, the root node
encodes $sig_i(key, 0)$ in a bit vector of length $2^i$,
with each bit corresponding to one of the $2^i$ combinations possible
among $i$ bits.  For example, where $i=4$, the 4-bit sequence {\tt 0000}
would correspond to the first of the vector's 16 elements.  This is illustrated
in Figure \ref{fig:example-vector}.

\paragraph{}
Just as the root node, with depth 0, encodes the sequence $sig_i(key, 0)$,
nodes at depth 1 encode the sequence $sig_i(key, 1)$, and,
in general, nodes at depth $d$ encode $sig_i(key, d)$.  Nodes at the
same depth, but which do not share a parent (i.e., ``cousin'' nodes)
encode the same subsequence of the key space, but necessarily differ
in the bit sequence which precedes the sequence they encode.  This
preceding sequence is the node's bit prefix.

\subsubsection{Vector Encoding}
An ``on" ({\tt 1}) bit in the root node's bit vector indicates the existence
of at least one key whose first $i$ bits match the $i$-bit combination
indicated by the bit vector.  Correspondingly, an ``off" bit ({\tt 0}) indicates
the absence of any such element.  For example, in the case $i=4$,
if the root node's bit vector is {\tt 1000000000000001}, then the tree is
guaranteed to contain at least one key $k_1$ with $sig_4(k_1, 0) = {\tt 0000}$
and at least one key $k_2$ with $sig_4(k_2, 0)={\tt 1111}$.

\subsubsection{Node Hierarchy}
Unlike typical tree structures, in which a parent node contains
pointers to its child nodes, nodes in this tree have no direct connection to
their child nodes.  Instead, a hash table called {\tt Node} is used for random-access to any
node in tree.  This is possible because every node in the tree (and, more strongly,
every possible node) can be uniquely specified by its bit-prefix.  For example,
the bit-prefix {\tt 00000000} specifies the left-most node at a depth of 2, and the
empty bit-prefix specifies the root (which is the initial position and has no prefix).

\subsubsection{Leaves}
Unlike other nodes, a materialized leaf node contains a table, which, for all keys
in the node's $2^i$ value range, maps each key to a list of object references.
This allows for the recovery of all associated objects, for example, the row of
which this key is a column-value or the class instance of which this key is a member
variable.



%%%%%%%%%%
% Operations
%%%%%%%%%%

\subsection{Supplemental Operations}

\begin{figure}[t]
\center
\fbox{
\includegraphics[width=3in]{insertion-05.pdf}
}
\caption{Insertion of a key.}
\label{fig:insertion-diagram}
\end{figure}

\subsubsection{Insertion}
Figure \ref{fig:insertion-diagram} outlines the insertion of the key
{\tt 111011010000000} along with the object of which this key
is a component.  As described in A, the full key is used with the node-hash
to access the proper leaf node, and a reference to the accompanying object
is appended to the list of objects registered to the key.  If the corresponding
bit is already on, then the insertion operation concludes. If the bit is off, then 
it is flipped, and the node's member count is incremented.  The leaf node's
parent is then accessed by splicing off the least-significant $i$-bits from the key.
It may be the case that this node does not yet exist, in which case an empty
node is initialized.  In either case, the parent's progeny count is incremented
and the bit corresponding to the leaf node is queried.  If the bit is already on,
then the process terminates; otherwise, the bit is flipped and this process advances
up the tree, terminating after reaching (and, if necessary, updating) the root node.

\subsubsection{Deletion}
Deletion requires both a key and an object as parameters, as multiple
objects may be registered to the same key.  The full bit-sequence of the
key is used with the node-hash to obtain its containing leaf node.
The list of objects registered to the key is obtained via the node's
reference table and is scanned for an object that matches the object parameter.
If a match is found, then the reference is removed.  The list is queried to determine
whether any objects remain registered to this key, if not, the bit corresponding to the
key is turned off, and the member count for the node is decremented.
If this update results in a member count of zero, then the node is decommissioned
and the tree's node hash for this key is set to null.

\paragraph{}
If the member count was decremented, then the parent node is loaded
and its member count is decremented.  If the child node was decommissioned,
then the bit corresponding to that child is turned off and the parent node's
progeny count is decremented.  As a bit change occurred, the vector is passed
as a parameter to the mapping function; if the mapping function returns an
empty list, that indicates that the node has no children remaining, and so it too
is decommissioned.  The process repeats for each parent node, up to and
including the root node.

\begin{figure}[b]
\fbox{
\includegraphics[width=6in]{intersection-06.pdf}
}
\caption{Initial step in the intersection of 3 sets}
\label{fig:intersection-diagram}
\end{figure}

\subsection{Intersection}
\ref{fig:intersection-diagram}
\subsubsection{Local Intersection}














%Intersect trees ��prune all branches that fail intersection
%Determine which subtrees to compare
	
	
%Compare subtrees
%Begin traversing from root
%	Among all m trees, does any have memberCount of q, where q is a parameter indicating the threshold for direct-value comparison
%		if so, find the tree least likely to contain a match in that range.  
%			fewest members
			

\section{Future Work}
					
Encodings that remap dense data to a sparse distribution

Better than 4-bit mappings




\section{Conclusion}

\section*{Acknowledgments}

We would like to thank Wes Filardo







\bibliographystyle{naaclhlt2010}
\bibliography{amt}

%\bigskip % This is to push the Appendix A section onto page 5. There's gotta be a better way.
%\bigskip



%\begin{figure}[h]
%\small
%\center
%\begin{tabular}{|p{70mm}|}
%\begin{enumerate}
%%\setlength{\itemsep}{0pt}
%\item The sentence expresses the relation. \\
%Sentence: For the past eleven years, James has lived in Tucson. \\
%$Relation$: ``Tucson'' is the residence of ``James''\\
%\item The sentence does not express the relation.\\
%\textit{Sentence}: Samuel first met Divya in 1990, while she was still a student.\\
%\textit{Relation}: ``Divya'' is a spouse of ``Samuel''\\
%\item The relation does not make sense.\\
%\textit{Sentence}: Soojin was born in January.\\
%\textit{Relation}: ``January'' is the birth place of ``Soojin'' \\
%\end{enumerate}
%\end{tabular}
%\caption{\label{fig:response-types} The three annotation options; enumerated
%with examples.}
%\end{figure}



%\subsection*{HIT Contents}
%
%The positive examples came from two sources.  For 
%\textit{Experiment 1}, we used positive examples from the ACE \shortcite{ace_2004}
%dataset. For \textit{Experiment 2}, we hand annotated examples from
%the Tac KBP09 corpus. Figure \ref{fig:contrast-positives} presents the
%contrast of these two types of positive examples.
%
%
%\begin{figure}
%\small
%\center
%\begin{tabular}{|p{70mm}|}
%\hline
%\textbf{TAC KBP 09 positive example:} \\ \hline
%\textit{Sentence:} There was a flurry of activity as actress Suzanne Somers and her husband, Alan Hamel, pulled up to the front of their burned-out house in a black Cadillac Escalade.\\
%\textit{Relation:} ``Suzanne Somers'' is/are a spouse of ``Alan Hamel.''\\ \hline
%\textbf{ACE 2004 positive example:} \\ \hline
%\textit{Sentence}: This afternoon the Soviet Union president's news spokesperson, Igor Najinko, attended the news conference. \\
%\textit{Relation}:  ``president'' has/have a business relationship with ``spokesperson.'' \\
%\hline
%\end{tabular}
%\caption{\label{fig:contrast-positives} Contrast of positive examples from the TAC and ACE corpora.}
%\end{figure}
%
%\subsection*{Results}
%
%Experiment 1 used the ACE 2004 dataset for positive examples. For
%these known examples with question type \textit{Expressed}, 35\% were
%labeled as \textit{Nonsense}, and 14\% were labeled as \textit{Not Expressed}. 
%Although we filtered out relations between pronoun
%entity mentions, we chose to include relations between nominal entity
%mentions. When composed into a sentence, these relations often lacked
%a determiner for the nominal entity mention or were otherwise
%expressed ungrammatically. We expect that these issues left the
%workers undecided about how to label such examples. Figure
%\ref{fig:contrast-positives} presents these issues through an example
%ACE sentence/relation pair.


\end{document}



